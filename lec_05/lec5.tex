\documentclass{beamer}

\usepackage[utf8]{inputenc}
\usepackage[english]{babel}
\usepackage{datetime}
\usepackage{hyperref}
\usepackage{listings}
\usepackage{xcolor}
\usepackage{soul}
\usepackage{fancyvrb}
\usepackage{amsmath}
\usepackage[font=small,labelfont=bf]{caption}


\graphicspath{ {./} }

\definecolor{codegreen}{rgb}{0,0.6,0}
\definecolor{codegray}{rgb}{0.5,0.5,0.5}
\definecolor{codepurple}{rgb}{0.58,0,0.82}
\definecolor{backcolour}{rgb}{0.95,0.95,0.92}

\lstdefinestyle{mystyle}{
    backgroundcolor=\color{white},   
    commentstyle=\color{codegreen},
    keywordstyle=\color{magenta},
    numberstyle=\tiny\color{codegray},
    stringstyle=\color{codepurple},
    basicstyle=\ttfamily\footnotesize,
    breakatwhitespace=false,         
    breaklines=true,                 
    captionpos=b,                    
    keepspaces=true,                 
    numbers=left,                    
    numbersep=5pt,                  
    showspaces=false,                
    showstringspaces=false,
    showtabs=false,                  
    tabsize=2
}

\lstset{style=mystyle}

\title{Asymmetric cryptography: RSA and Rabin cryptosystems}
\author{Alexander Buchnev}

\newdateformat{monthYear}{\monthname[\THEMONTH] \THEYEAR}
\date{\monthYear\today}

\DeclareCaptionFormat{custom}
{%
    \textbf{\tiny #1#2}\textit{\tiny #3}
}
\captionsetup{format=custom}

\begin{document}

\frame{
	\titlepage
}

\section{Terminology reminder}

\begin{frame}
    Friendly terminology reminder
\end{frame}

\begin{frame}{Formal definition of PKC}
    \section{Formal definition of Public Key Cryptography}
    Public Key Cryptography consists of the public key, private key, 
    the encryption algorithm and the decryption algorithm. Let us see how these 
    things interact with each other.
\end{frame}

\begin{frame}{Formal definition of PKC}
    \begin{definition}[Public Key Cryptography]
        Let $\kappa$ be the security parameter of the cryptosystem. (note that 
        $\kappa$ is not necessarily the key length) An encryption scheme is then 
        defined by the following spaces (all depending on the security parameter 
        $\kappa$):
        \begin{itemize}
            \item $M_\kappa$ --- Space of all possible messages
            \item $PK_\kappa$ --- Space of all public keys
            \item $SK_\kappa$ --- Space of all secret keys
            \item $C_\kappa$ --- Space of all ciphertexts
            \item $KeyGen$ --- \textbf{Randomized} algorithm that takes $\kappa$ as input and generates
            public and private key in polynomial time. 
            \item $Encrypt : M_\kappa \times PK_\kappa \to C_\kappa$ 
            \item $Decrypt : C_\kappa \times SK_\kappa \to M_\kappa$
        \end{itemize}
    \end{definition}
\end{frame}

\section{EASY cryptosystem}

\begin{frame}{EASY cryptosystem example}
    Imagine we have the following cryptosystem, let us call it EASY:
    \begin{flalign*}
        & m, n \in \mathbb{Z},\ \gcd(p, n) = 1\\
        & PK := (p, n) \\
        & SK := (d, n), \text{ where } d = p^{-1} \pmod n \\ 
        & Encrypt := m \cdot p \pmod n \\
        & Decrypt := c \cdot d \pmod n
    \end{flalign*}
    \begin{example}
        \begin{flalign*}
            & PK = (3, 11) \quad SK = (4, 11) \\
            & Encrypt := m \cdot 3 \pmod{11} \\    
            & Decrypt := c \cdot 4 \pmod{11}
        \end{flalign*}
    \end{example}
    What are the $PK_\kappa$, $SK_\kappa$ and $C_\kappa$ for this cryptosystem?
\end{frame}

\begin{frame}
    \begin{flalign*}
        & m, n \in \mathbb{Z},\ \gcd(p, n) = 1\\
        & SK := (d, n) \\ 
        & PK := (p, n), \text{ where } d = p^{-1} \pmod n \\
        & Encrypt := m \cdot p \pmod n \\
        & Decrypt := c \cdot d \pmod n
    \end{flalign*}
    The main question is: how to recover the private key if we know the public 
    one?
\end{frame}

\begin{frame}{EASY cryptosystem cryptanalysis}
    The answer is rather simple and lays directly in the specification, indeed, 
    to obtain the public key we use the inverse modulo operation 
    $p = d^{-1} \pmod n$. We can use the same operation, to obtain the private 
    key! $d = p^{-1} \pmod n$
\end{frame}

\begin{frame}{How fast is this attack?}
    \pause
    The value $d = p^{-1} \pmod n$ can be computed fast: using the fast powering
    algorithm one can compute it with the following asymptotic: 
    $\mathcal{O}(\log n)$.
\end{frame}

\section{RSA cryptosystem}

\begin{frame}{RSA cryptosystem}
    The RSA cryptosystem:
    \begin{flalign*}
        & p, q \overset{\$}{\leftarrow} \mathbb{Z} \\
        & n := p \cdot q,\quad e \overset{\$}{\leftarrow} \mathbb{Z} : \gcd(n, e) = 1 \\
        & d := e^{-1} \pmod {(p - 1) (q - 1)}\\
        & PK := (e, n) \\
        & SK := (p, q, d) \\ 
        & Encrypt := m^e \pmod n \\
        & Decrypt := c^d \pmod n
    \end{flalign*}
\end{frame}

\begin{frame}[allowframebreaks]{How does the RSA cryptosystem work?}
    First, let us note the following thing: \\
    $d := e^{-1} \pmod {(p - 1) (q - 1)}$ and 
    $ed \equiv 1 \ \pmod {(p - 1) (q - 1)}$ is the same thing. Having this, we 
    can rewrite the equation above in the following manner:
    $ed = 1 + k (p - 1) (q - 1), k \in \mathbb{Z}$, we shall use it later.

    Now we want to prove that the following holds for the RSA:
    \begin{flalign*}
        & m \in M_\kappa, pk \in PK_\kappa, sk \in SK_\kappa \\ 
        & Decrypt(Encrypt(m, pk), sk) = m
    \end{flalign*}

    Let us instantiate with our $Encrypt$ and $Decrypt$:
    \begin{flalign*}
        & m \in M_\kappa, pk \in PK_\kappa, sk \in SK_\kappa \\ 
        & m \equiv (m^e)^d \pmod n \Rightarrow \\
        & m \equiv m^{ed} \pmod n \Rightarrow \\
        & m \equiv m^{1 + k (p - 1) (q - 1)} \pmod n \Rightarrow \\
        & m \equiv m \cdot (m^{(p - 1) (q - 1)})^k \pmod n \Rightarrow
    \end{flalign*}

    Let us recall, what is the value for the Euler's $\varphi$ function for the 
    $n$ in the form $n = pq$, where $p$ and $q$ are primes?
    
    It is exactly $\varphi(n) = \varphi(pq) = (p - 1)(q - 1)$.

    Then, by the Euler's Theorem, we obtain the following:
    \begin{flalign*}
        & m \in M_\kappa, pk \in PK_\kappa, sk \in SK_\kappa \\ 
        & m \equiv m \cdot (m^{(p - 1) (q - 1)})^k \pmod n \Rightarrow \\
        & m \equiv m \cdot (1)^k \pmod n \Rightarrow \\
        & m \equiv m \pmod n \qed
    \end{flalign*}

\end{frame}

\begin{frame}{The RSA cryptosystem cryptanalysis}
    One of the ways to break the cryptosystem is to recover the private key from 
    the public one, as we did it in EASY cryptanalysis, but in that case it was 
    indeed easy, it is not the case for the RSA...
\end{frame}

\begin{frame}{A way to break the RSA}
    The \textbf{trapdoor} function in RSA is the integer powering with the 
    composite modulus $n = pq$. If we know how to factor $n$, we can easily
    compute the decryption constant $d = e^{-1} \pmod {(p - 1) (q - 1)}$, thus
    we obtain the ability to decrypt all messages.
\end{frame}

\begin{frame}[fragile]{Python implementation of RSA}
    \texttt{using pycryptodome (pip install pycryptodome)}
    \begin{lstlisting}[language=Python]
#!/bin/env python

import Crypto.Util.number

e = 2 ** 16 + 1

# security parameter
N_bits = 2048

p = number.getStrongPrime(N_bits, e)
q = number.getStrongPrime(N_bits, e)

d = (p - 1) * (q - 1)
n = p * q

pub_key = (e, n)
priv_key = (d, n)

        \end{lstlisting}
\end{frame}

\begin{frame}[allowframebreaks]{Fermat's attack}
    As we remember, in the RSA cryptosystem, the public key modulus has the form
    $n = pq$, where $p$ and $q$ are primes. Later in this lecture we will 
    discuss one of the ways to break the RSA cryptosystem with weak primes, 
    using the Fermat's attack.
    
    Pierre de Fermat discovered the way to factor numbers of the form $n = pq$
    in the 17th century.  

    Method description:
    Since $n$ is odd, then both $p$ and $q$ are also odd. It is stated that one
    can make the following substitution:
    \begin{eqnarray*}
        & a = \frac{p + q}{2}, b = \frac{p - q}{2} \\
        & N = a^2 - b^2 = \left(\frac{p + q}{2}\right)^2 - 
                          \left(\frac{p - q}{2}\right)^2 = pq
    \end{eqnarray*}

    Without loss of generality let us assume that $p < q$ and they are 
    \textbf{relatively close} to each other. In that case, if we find $a$ such 
    that the left part of the equation is the perfect square, then we managed to
    factor the $n$!!

    And as for the starting values, let us take 
    $a = \left\lceil \sqrt{N} \right\rceil$ and continuously increase $a$ by $1$
    each time we did not guess.
\end{frame}

\begin{frame}[fragile]{Pseudocode implementation}
    \begin{lstlisting}[language=Python]
def fermat_factorisation(N):
    a = ceil(sqrt(N))
    b2 = a ^ 2 - N
    
    while not b2.is_square():
        a += 1
        b2 = a ^ 2 - N
    
    return a - sqrt(b2)
    \end{lstlisting}
\end{frame}

\begin{frame}{Other attacks}
    At the same time, this was only one of \textbf{MANY} attacks on RSA, the 
    non-exhaustive list of attacks with vulnerabilities about RSA is presented 
    below:
    \begin{itemize}
        \item Common modulus attack
        \item Coppersmith's attack
        \item Small $d$ attack
        \item Low public exponent attack
        \item Håstad's broadcast attack
        \item etc.
    \end{itemize}
\end{frame}

\begin{frame}{Conclusions}
    \begin{itemize}
        \item Is RSA bad?
        \item Why so many number-theoretic attacks?
        \item Maybe better PRNG?
        \item Quantum computers?
    \end{itemize}
\end{frame}

\end{document}
