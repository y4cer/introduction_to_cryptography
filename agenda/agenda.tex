\documentclass{article}
\usepackage[english,russian]{babel}
\usepackage[utf8x]{inputenc}
\usepackage{amssymb}
\usepackage{amsmath}
\usepackage{geometry}
\usepackage{eucal}
\usepackage{cite}
\usepackage{rotating}
\usepackage{booktabs}
\usepackage{multirow}
\usepackage{lscape}
\usepackage[table,xcdraw]{xcolor}
\usepackage[font=small,labelfont=bf]{caption}
\usepackage{listings}
\usepackage{xcolor}
\usepackage{float}
\usepackage[hidelinks]{hyperref}
\usepackage[inkscapeformat=png]{svg}


\geometry{
    a4paper,
    left=20mm,
    top=10mm,
    bottom=15mm,
}

\title{Introduction to cryptography (course)}
\author{Alexander Buchnev}
\date{2023}
\graphicspath{ {./} }

\begin{document}
\maketitle

\section*{Introduction}
So, the time has come, I finally decided to create an introductory course on cryptography and read it during Fall 2023 
semester at Innopolis University. Approximately, course will consist of 10 lectures + corresponding homeworks, and if I 
can manage it, I'll try to create a some sort of final exam. 

\section*{Course prerequisites}
There are no specific course prerequisites, although it would be great if students know some mathematical analysis, 
linear algebra and set theory, as well as can read the mathematical formulas.

\section*{Course description}
This course is for those who want to become somewhat familiar with cryptography and for those who want to flex their 
brain muscles with new concepts. It is okay if you don't know much about cryptography and why is it important, at the 
end of the course you will be able to factor numbers of order $2^{1024}$ (joke) and hack some of weak implementations of
cryptographic primitives (not a joke). The course consists of 10 lectures and homeworks with practical applications to 
real life. We will also consider some real-life attacks and some silly ones, from CTFs. 

\section*{Course outcomes}
The student is expected to become familiar with basic algebraic notations and definitions, as types of morphisms, 
semigroups, monoids, groups and rings, as well as obtain some practical experience with cryptanalysis and applied 
cryptography. 

\section*{Course structure}
\begin{enumerate}
    \item Introduction + Basic Abstract algebra with SageMath
    \item Basic abstract algebra with SageMath (cnt.)
    \item Public key cryptography + concept of secure messaging % OTP
    \item Message integrity: secure hash functions and random numbers
    \item Asymmetric cryptography: RSA and Rabin cryptosystems
    \item Asymmetric cryptography: DHKE with DLP some other key-exhange protocols 
    \item Asymmetric cryptography: ECDHKE and TLS/SSL
    \item Asymmetric cryptography: digital signatures, GOST and Bitcoin curves
    \item Inroduction to symmetric cryptography: block ciphers
    \item Inroduction to symmetric cryptography: stream ciphers
\end{enumerate}

\end{document}