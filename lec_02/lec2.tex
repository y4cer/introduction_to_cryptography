\documentclass{beamer}

\usepackage{array}
\usepackage[english]{babel}
\usepackage[utf8x]{inputenc}
\usepackage{amssymb}
\usepackage{amsmath}
\usepackage{geometry}
\usepackage{graphicx}
\usepackage{eucal}
\usepackage{cite}
\usepackage{datetime}
\usepackage{soul}
\usepackage[font=small,labelfont=bf]{caption}


\graphicspath{ {./} }

\title{Introduction to (mathematical) cryptography}
\subtitle{Basic abstract algebra with SageMath (cnt.)}
\author{Alexander Buchnev}

\newdateformat{monthYear}{\monthname[\THEMONTH] \THEYEAR}
\date{\monthYear\today}

\begin{document}

\frame{
	\titlepage
}

\newtheorem{prop}{Proposition}

%% Proof for the homework

\begin{frame}{Outline}
    \section{Outline}
	\tableofcontents
\end{frame}

\begin{frame}{Semigroup}
    \section{Algebraic structures}
	\subsection{Semigroup}
	\begin{definition}
		A semigroup $S$ is a set $\mathcal{S}$ together with binary operation $+$ that satisfy the following property:
		\begin{itemize}
			\item $S$ is closed under $+$, i.e. for all $a, b \in S$ \\
				$a + b \in S$
			\item Associativity, i.e. for all $a, b, c \in S$ \\
				$(a + b) + c = a + (b + c)$
		\end{itemize} 
	\end{definition}

	\begin{example}
		\begin{itemize}
			\item Empty semigroup $S = (\{\}, +)$
			\item Semigroup with exactly 1 element. $S = (\{a\}, \cdot)$, where $a \cdot a = a$
			\item Binary strings (strings in general). $S = (\{0, 1\}^*, +)$.
			\item Natural numbers under addition. $S = (\mathbb{N}, +)$
		\end{itemize}
	\end{example}
\end{frame}

\begin{frame}{Monoid}
	\subsection{Monoid}
	% It is important that the identity is defined as two-sided.
	\begin{definition}
		A monoid $M$ is a semigroup $S$, which satisfies additional property:
		\begin{itemize}
			\item $\exists \varepsilon \in M : a + \varepsilon = \varepsilon + a = a$
		\end{itemize}
	\end{definition}

	\begin{example}
		\begin{itemize}
			\item Natural numbers under multiplication (identity is $1$). $M = (\mathbb{N}, \cdot)$.
			\item Natural numbers under addition (identity is $0$). $M = (\mathbb{N}, +)$.
			\item Binary strings (identity is an empty string). $S = (\{0, 1\}^*, +)$.
		\end{itemize}
	\end{example}
\end{frame}

\begin{frame}{Monoid morphisms}
	\subsection{Monoid morphisms}
	\begin{definition}
		Monoid homomorphism is a structure preserving map $\varphi : (M, +) \to (L, \star)$, such that for all 
		$m, n \in M$ the following holds: $\varphi(m + n) = \varphi(m) \star \varphi(n)$
	\end{definition}
\end{frame}

\begin{frame}
	\begin{definition}[Monomorphism]
		A group homomorphism that is injective (or, one-to-one); i.e., preserves distinctness.
	\end{definition}
	\begin{definition}[Epimorphism]
		A group homomorphism that is surjective (or, onto); i.e., reaches every point in the codomain.		
	\end{definition}
	\begin{definition}[Isomorphism]
		A group homomorphism that is bijective; i.e., injective and surjective. Its inverse is also a group 
		homomorphism.	
	\end{definition}
	\begin{definition}[Endomorphism]
		A group homomorphism, $h: G \to G$; the domain and codomain are the same. Also called an endomorphism of $G$.		
	\end{definition}
\end{frame}

\begin{frame}{Group}
	\subsection{Group}
	% groups are the central topic of the group theory (interestengly)
	\begin{definition}
		A (abelian/commutative) group $G$ us a monoid $M$, which satisfies additional property (properties):
		\begin{itemize}
			\item $\exists a^{-1} \in G : a + a^{-1} = \varepsilon$
			\item[(abelian)] for all $a, b \in G\ a + b = b + a$
		\end{itemize}
	\end{definition}
	\begin{definition}
		Order of a finite group is a number of elements in this group. If the group is infinite, one says that the order
		is infinite.
	\end{definition}
\end{frame}
	
\begin{frame}{Group examples}
	\begin{example}[Groups]
		\begin{enumerate}
			\item Empty group. $G = (\{\}, +)$
			\item Real numbers under addition $G = (\mathbb{R}, +)$.
			\item Matrix multiplication with nonzero determinants.
		\end{enumerate}
	\end{example}
	\begin{example}[Non-groups]
		\begin{itemize}
			\item Natural numbers under addition or multiplication. $G = (\mathbb{N}, +)$.
			\item Integers mod 10 under multiplication (0 has no inverse).
		\end{itemize}
	\end{example}
\end{frame}

\begin{frame}{Subgroups}
	\subsubsection{Subgroups}
	\begin{definition}[of subgroup]
		$H$ is a subgroup of $G$ iff $H$ is a non-empty subset of $(G, \star)$ which is closed under the binary 
		operation on $G$ and is closed under inverses, i.e. $\forall h, k, \in H, \ hk$ and $h^{-1} \in H$. 
		
		The	subgroups are denoted as $H \le G$. If $H \le G$ and $H \ne G$ we shall write $H < G$.
	\end{definition}
	\begin{prop}
		All subgroups of abelian groups are abelian.
	\end{prop}
	\begin{example}
		\begin{itemize}
			\item $G = \mathbb{Z}, H = (2\mathbb{Z}, +)$ \newline
			where $n\mathbb{Z} = \{\ldots, -2n, -n, 0, n, 2n, \ldots\}$
			\item $(G, \star)$ - any group, $H = (\{\varepsilon_G\}, \star)$
		\end{itemize}
	\end{example}
	Any group has at least 2 trivial subgroups: containing identity element and the group itself.
\end{frame}

\begin{frame}{Group generators}
	\subsubsection{Group generators}
	\begin{definition}
		Given $(G, \cdot)$ - the ``multiplicative`` group. \newline
		Let us denote $\underbrace{a \cdot a \cdot \ldots \cdot a}_{n\ times} = a^n$, and \newline
		$a^{-n} = \underbrace{a^{-1} \cdot a^{-1} \cdot \ldots \cdot a^{-1}}_{n\ times}$.
	\end{definition}
	\begin{definition}[subgroups generated by subsets]
		$A$ is a subset of $G$.

		$\overline{A} = \{a_1^{\epsilon_1},\ldots, a_n^{\epsilon_n}\ \vert\ n \in \mathbb{Z}$ and $a_i \in A,\ 
		\epsilon_i = \pm 1$ for each $i \}$
		
		$\overline{A} = \langle A \rangle$ 
	\end{definition}
	$\langle A \rangle \le G$
	\begin{example}
		$G = \mathbb{Z}, H = (\langle 2 \rangle, +)$
	\end{example}
	\begin{definition}
		Order of element $e$ in group $G$ is the order of the subgroup generated by the element.
	\end{definition}
\end{frame}

\begin{frame}{Normal subgroups}
	\subsubsection*{Normal subgroups}
	\begin{definition}
		A subgroup $N$ of $G$ is called normal if and only if it is invariant under conjugation. \newline
		$N \unlhd G$ \newline
		$g n g^{-1} \in N$ for all $g \in G$, or equivalently \newline 
		$g N g^{-1} = N$ for all $g \in G$.
	\end{definition}
	\begin{prop}
		All subgroups of abelian groups are normal.
	\end{prop}
	\begin{example}
		\begin{itemize}
			\item $G$ - any group, $N = \{\varepsilon\}$
			\item $G = \mathbb{Z}, N = n\mathbb{Z}$
		\end{itemize}
	\end{example}
	$\triangle$
\end{frame}

%%%%%%%%%%%%%%%%%%%%%%%%%%%%%%%%%%%%%%%%%%%%%%%%

\begin{frame}{Cosets and Lagrange's theorem}
	\section{Cosets and Lagrange's theorem}

	\begin{definition}[Coset]
		Given a group $G$ and $H$ is a subgroup of $G$. The left cosets of $H$ in $G$ are: \newline
		$gH = \{gh : h \in H\} \text{ for all } g \in G$ \newline
		Right cosets are defined by analogy: \newline
		$Hg = \{hg : h \in H\} \text{ for all } g \in G$
	\end{definition}
	
	\begin{theorem}[Lagrange's theorem]
		Given a group $G$ and $H$ is a subgroup of $G$, order of $H$ divides order of $G$.
	\end{theorem}
	\begin{example}
		$G = (\mathbb{Z} / 15\mathbb{Z}, +), H = (\mathbb{Z} / 3\mathbb{Z}, +)$ \newline
		$\# H \mid \# G$, since $3 \mid 15$. 
	\end{example}
	
\end{frame}

% https://en.wikipedia.org/wiki/Coset
\begin{frame}{Proof of Lagrange's theorem}
	\begin{proof}[Proof (Lagrange's theorem)]
		\begin{enumerate}
			\item Any two left cosets have the same cardinality
			\begin{proof}[Proof (1)]
				Suppose there exists coset $aH, a \in G$, one can define bijection $h \mapsto ah, h \in H$.
			\end{proof}
			\item $a, b \in G$, and $aH$ and $bH$ are cosets. If $aH$ and $bH$ share at least 1 element, they coincide
			\begin{proof}[Proof (2)]
			Suppose there exists a shared element $ax = by, x,y \in H$, then $a = byx^{-1}$, but $yx^{-1} \in H$, hence
			$aH = byx^{-1}H = bH$
			\end{proof}
		\end{enumerate}
		Hence we conclude that $G$ is the disjoint union of its left cosets of $H$ and each left coset has the same 
		cardinality. \newline
		$\Rightarrow \# H \mid \# G$
	\end{proof}
\end{frame}

\begin{frame}{Euler's theorem}
	\section{Euler's theorem}
	\begin{Theorem}[Euler's theorem]
		\begin{align*}
			& a \in \mathbb{Z}, n \in \mathbb{Z}_{>0}, (a, n) = 1 \\
			& a^{\varphi(n)} \equiv a \pmod n, \text{ where } \\
			& \varphi(n) = \# \mathbb{Z} / n \mathbb{Z}^*
		\end{align*}
    \end{Theorem}
	First, let us look at the set $\mathbb{Z} / n \mathbb{Z}^*$. It is a multiplicative group modulo $n$, i.e. 
	each element of group $\mathbb{Z} / n \mathbb{Z}^* = \left\{ x : (x, n) = 1 \right\}$ has an inverse modulo $n$.
\end{frame}

\begin{frame}{Proof of Euler's theorem}
	\begin{proof}[Proof (Euler's theorem)]
		If $a$ is coprime to $n$, then it must be in some residue class of $n$. Then 
		$\langle a \rangle \le \mathbb{Z} / n \mathbb{Z}^*$. \newline
		By Lagrange's theorem, $k = \#\langle a \rangle \ | \ \#\mathbb{Z} / n \mathbb{Z}^* \Rightarrow \varphi = kM$. 
		\newline
		Then, $a^{\varphi(n)} = a^{kM} = (a^{k})^M = 1^M \equiv 1 \pmod n$ 
	\end{proof}
	\begin{proof}[Proof (Fermat's little theorem)]
		And Fermat's little theorem is just a particular case of Euler's theorem. \newline
		$a ^ {p - 1} = a ^ {\varphi(p)} \equiv 1 \pmod p$, where $\varphi(p) = p - 1$ 
	\end{proof}
\end{frame}

\begin{frame}{Quotient groups (proof is left as an optional exercise)}
	\section{Quotient groups}
	\begin{definition}ement $g \in G$ and a normal subgroup $N$ of $G$, one can define $G / N$ the set of 
		all cosets of $N$ in $G$. That is: $G / N = \{ gN, \text{for all } g \in G\}$ is called the quotient group.
	\end{definition}

	\begin{theorem}
		If $N \unlhd G$ then $Q = N/G$ forms a group known as a quotient group with the following group operation: 
		\newline
		$(xN)(yN) = (xy)N, \forall x, y \in G$.
	\end{theorem}
\end{frame}

\begin{frame}{Discrete logarithm problem in $\mathbb{Z}^*_n$}
	\section{Discrete logarithm problem in $\mathbb{Z}^*_n$}
	%since $\mathbb{Z}^*_n$ indeed forms a group, we can construct the discrete log problem
	\begin{block}{}
		For given $n \in \mathbb{N}_+$ the discrete logarithm problem in $\mathbb{Z}^*_n$ can be formulated the 
		following way:
	\end{block}
	
	\begin{block}{}
		For any $a, b \in \mathbb{Z}^*_n$ find $x : a^x = b \mod n$. If such $x$ exists, it is called the discrete 
		logarithm of $b$ with respect to base $a$.
	\end{block}
	
	\begin{example}[DLP in $\mathbb{Z}^*_{13}$]
		\begin{tabular}{>{\small}c>{\small}c>{\small}c}
			$2^0 = 1$ & $2^4 = 3$ & $2^8 = 9$ \\
			$2^1 = 2$ & $2^5 = 6$ & $2^9 = 5$ \\
			$2^2 = 4$ & $2^6 = 12$ & $2^{10} = 10$ \\
			$2^3 = 8$ & $2^7 = 11$ & $2^{11} = 7$
		\end{tabular}
	\end{example}
	% on the whiteboard example when DLP does not have any solutions
\end{frame}

\begin{frame}{Homework}
    \section{Homework}
	Let $n, m \in \mathbb{N} \backslash \{0\} : (n, m) = 1$. \newline
	Build a group isomorphism 
	$\psi : \mathbb{Z} \to \mathbb{Z} / n \mathbb{Z} \times \mathbb{Z} / m \mathbb{Z}$
\end{frame}

\begin{frame}{Next lecture}
    \section{Next lecture topic}
	Next lecture will be devoted to essential cryptographical concepts such as public-key cryptography (finally!) as 
	well as some information theoretical concepts. 
\end{frame}

\begin{frame}{Suggested topics to read}
    \section{Suggested topics to read}
    \begin{itemize}
		\item Permutation groups, dihedral groups
		\item Sylow p-groups
		\item The Grothendieck group (interesting construction!)
	\end{itemize}    
\end{frame}

\end{document}