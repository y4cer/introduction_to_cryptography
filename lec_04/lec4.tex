\documentclass{beamer}

\usepackage{array}
\usepackage[english]{babel}
\usepackage[utf8x]{inputenc}
\usepackage{amssymb}
\usepackage{amsmath}
\usepackage{geometry}
\usepackage{graphicx}
\usepackage{eucal}
\usepackage{cite}
\usepackage{wrapfig}
\usepackage{datetime}
\usepackage[font=small,labelfont=bf]{caption}

\graphicspath{ {./fig} }
\setbeamertemplate{bibliography item}{\insertbiblabel}

\title{Message integrity: secure hash functions and random numbers}
\author{Alexanar Buchnev}

\newdateformat{monthYear}{\monthname[\THEMONTH] \THEYEAR}
\date{\monthYear\today}

\DeclareCaptionFormat{custom}
{%
    \textbf{\tiny #1#2}\textit{\tiny #3}
}
\captionsetup{format=custom}

\begin{document}

\frame{
	\titlepage
}

\newtheorem{prop}{Proposition}

%% Proof for the homework

\section*{Outline}

\begin{frame}{Outline}
	\tableofcontents
\end{frame}

\begin{frame}{Hash functions}
    \section{What is the hash function?}
    \begin{definition}[Hash function]
        Hash function (one way function, cryptographic hash function) is a
        special type of function, which maps the arbitrary sized messages to
        the fixed size values.
    \end{definition}
    \begin{example}[Simple examples of hash function]
        \begin{itemize}
            \item reduction modulo $n$
            \item polynomial addition of string characters and then reduction
            modulo $p$
            \item byte operations (most common)
            % \item hash functions built on top of encryption schemes
        \end{itemize}
    \end{example}
\end{frame}

\section{Cryptographic hash function}

\begin{frame}{Cryptographic hash function}
    What is the cryptographic hash function?
    \pause
    \begin{definition}[Cryptographic hash function]
        Cryptographic hash function is a hash function which satisfies the
        following properties:
        \begin{itemize}
            \item \textbf{Preimage resistance.} Given the hash value $h$ is must be
            infeasible to find such message $m$ that hashes to $h$.
            \pause
            \item \textbf{Second pre-image resistance.} Given an input $m1$ it
            must be infeasible to find different $m2$ such that it has the
            same hash value.
            \pause
            \item \textbf{Collision resistance.} Given nothing other than the
            hash function itself, find 2 distinct inputs which will produce the
            same hash value.
        \end{itemize}
    \end{definition}
\end{frame}

\begin{frame}{Other properties of hash functions\footnote{https://crypto.stackexchange.com/questions/55435/designing-a-hash-function-from-first-principles-rather-than-depending-on-heurist}}
    \begin{itemize}
        \item \textbf{Strict avalanche criterion.} If you change any bit (or
        any sequence of bits), in the input, all bits in the output must have
        50\% change to change too.
        \pause
        \item \textbf{Nonlinearity.} If the hash function is linear, one can
        create the system of the linear equations and just solve it for the
        arbitrary preimages. Take the already mentioned reduction modulo $n$
        for example. Let us modify it a bit, so the resulting equation looks
        like this: $f(m) = a \cdot m + b \pmod n$. Then if we are given the
        hash value $f(m)$, we can easily construct the message by solving the
        equation: $m = a^{-1} \cdot (f(m) - b) \pmod n$. Although it is a toy
        example, one can come up with more hard, but linear one, which is prone
        to this method.
        \pause
        \item \textbf{Another thing to consider is balance}. You want your hash behave
        like the so called random oracle.
    \end{itemize}
\end{frame}

\section{Applications of (cryptographic) hash functions}

\begin{frame}{Applications of (cryptographic) hash functions}
    \begin{itemize}
        \item Message integrity check
        \item Message authentification
        \item Hash Tables
        \item PoW (Proof of Work) blockchain networks
        \item etc.
    \end{itemize}
\end{frame}

\section{Cryptographic hash function cosntructions}

\begin{frame}{How can one construct the hash function?}
    Below I am going to talk about 2 constructions: Merkle-Damgård and the
    Sponge construction. I chose these two because SHA1 and SHA2 were the NIST
    standard functions that are based on the Merkle-Damgård construction, and
    the SHA3 finalist (Keccak) is based on the Sponge construction, which is
    completely different from the Merkle-Damgård and it is worth discussing
    therefore.
\end{frame}

\subsection{Merkle-Damgård construction}

\begin{frame}{Merkle-Damgård construction}
    Some of the most used hash functions are based on Merkle-Damgård
    construction:
    \begin{itemize}
        \item MD5
        \item SHA1, SHA2
    \end{itemize}
\end{frame}

\begin{frame}{Step-by-step description of the Merkle-Damgård construction}
    \begin{enumerate}
        \item Pad the initial message with the appropriate padding scheme
        (in case of md5, it is 0x80 and a bunch of zeroes)
        \item Initialize the IV (input vector --- initial state)
        \item Process message in blocks
        \item Finalize
    \end{enumerate}
\end{frame}

\begin{frame}{Merkle-Damgård construction}
    \includegraphics[width=\textwidth]{Merkle-Damgard_hash_big.svg.png}
    \captionof*{figure}{General Merkle-Damgård construction scheme. wikipedia.org}
\end{frame}

\subsection{Sponge construction}

\begin{frame}{Sponge construction}
    \par According to final version of FIPS 202 where the SHA3 is descibed 
    (2015, 7 years after the initial publication of the Keccak algorithm),
    SHA3 is the family of the functions on binary data. Each version of SHA-3
    is based on the instance of the Keccak algorithm (family of the functions).
    Keccak is based on the Sponge construction.
    \newline \newline
    \pause
    \par In 2008 Guido Bertoni and his team introduced the Sponge Construction. Simply
    said, sponge function can take a message of the arbitrary length and
    produce the hash digest of the desired length.
    \newline \newline
    \pause
    \par There are two phases of the Sponge function. First is called `absorbtion',
    when the message is fed into the hash function, the second is when we
    `squeeze' the result.
\end{frame}

\begin{frame}{Sponge Construction}
    \includegraphics[width=0.8\textwidth]{SpongeConstruction.svg.png}
    \captionof{figure}{The Sponge Construction. wikipedia.org}
    Step by step process of absorbing phase:
    \begin{enumerate}
        \item SHA3 standard specifies the $r+c$ is $1600$
        \item Input message is padded to the multiple of $r$
        \item Input message is split into the $n$ block with each of length $r$
        \item For each input block we xor it with the current state of $r$,
        then we apply \textbf{permutation function $f$} to the whole state
    \end{enumerate}
\end{frame}

\begin{frame}{Squeezing phase}
    In the squeezing phase, we do the same operations, except for the input
    message part, i.e. we apply the \textbf{permutation function $f$} to the
    state and get the first $r$ bits of the state as the message digest.
\end{frame}

\section{General attacks on hash primitives}

\begin{frame}{General attacks on hash primitives}
    \begin{itemize}
        \item Rainbow Tables (bruteforce)
        \item Birthday Attack
        \item Differential Cryptanalysis
        \item SAT Cryptanalysis
    \end{itemize}
\end{frame}

\section{Salts, passwords and databases}

\begin{frame}{Salt}
    \begin{definition}[Cryptographic salt]
        Salt is a (random) piece of data which is appended to the message to
        create the different hash digest due to the strict avalanche criterion
        described above.
    \end{definition}
    \pause
    What are the benefits of `salting the passwords?'
    \pause
    The main benefit is that the salting helps protecting against the `rainbow
    table attacks', as well as protects passwords that occur multiple times in
    a database, as a new salt is used for each password.
\end{frame}

\section{Random number generators}

\begin{frame}{PRNG}
    \begin{definition}[PRNG - PseudoRandom Number Generator]
        PRNG (also known as Determenistic Random Number Generator --- since 
        everything is dependent on the initial \texttt{seed} value) is an 
        algorithm which generates the sequence of numbers with properties 
        approximate to the random numbers.
    \end{definition}
    \pause
    For now, there are no `perfect' PRNGs!!
\end{frame}

\begin{frame}{Ways to build the PRNG}
    \begin{itemize}
        \item Use block or stream cipher
        \item Use some hard mathematical assumption (e.g. Blum-Blum-Shub 
        algorithm, see details below)
        \item The Goldreich-Levin Theorem says that if you have perfect OWF, 
        then you can build the PRNG, but this is not used much because of the 
        speed issues
        \item Use the external source of entropy (some system interrupts, disk
        access timings, keyboard presses, etc.)
        \item Reinvent the wheel!! (you can put a `clown' emoji here (: )
    \end{itemize}
\end{frame}

\subsection{Blum-Blum-Shub algorithm}

\begin{frame}{Blum-Blum-Shub PRNG}
    The discription is rather simple in this case:
    \begin{flalign*}
        & x_{n+1} = x^2_n \pmod M,\\ 
        & M = pq \\
        & \text{seed is } x_0 : (x_0, M) = 1
    \end{flalign*}
    This PRNG is based on factorisation hardness assumption, like the RSA 
    cryptosystem.
\end{frame}

\subsection{Linear Congruental Generator (LCG)}

\begin{frame}{Linear Congruental Generator (LCG)}
    This is an example of bad design of the cryptographic PRNG, we shall 
    demonstrate its flaws on practice:

    And again, the recursive definition is helping us:
    \begin{definition}[LCG]
        \begin{flalign*}
            & x_{n+1} = a \cdot x_n + c \pmod m, \\
            & a - \text{multiplier}, \\
            & c - \text{increment}, \\ 
            & m - \text{big prime, aka modulus}
        \end{flalign*}
    \end{definition}
\end{frame}

% TODO: add examples with runs

\begin{frame}[allowframebreaks]{Why is this design bad for the cryptography?}
    One can rearrange the terms in a fancy manner and obtain the formula for the 
    $n+k$-th term, whre the $n$-th term is known:

    \begin{flalign*}
        & x_{n + k} \equiv a^k x_n +\frac{a^k - 1}{a - 1} c \pmod m \\
        & a \ge 2,\ k \ge 0
    \end{flalign*}

    And, while you might think, that this is not an achievement, this formula 
    gives us \textbf{any} term of the sequence given the initial values at the 
    $O(\log(k))$ time and $O(1)$ space complexity. 
\end{frame}

\begin{frame}
    Its Jupyter Notebook time!
\end{frame}

\begin{frame}{Homework}
    Implement the LCG and the corresponding attack that we have covered today
\end{frame}

% \begin{frame}[allowframebreaks]{References}
% 	\section*{References}
% 	\bibliography{refs}
% 	\bibliographystyle{ieeetr}
% \end{frame}

\end{document}
